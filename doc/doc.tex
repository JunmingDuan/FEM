\documentclass[a4paper, 11pt]{ctexart}
\usepackage{srcltx,graphicx}
\usepackage{amsmath, amssymb, amsthm}
\usepackage{color}
\usepackage{lscape}
\usepackage{multirow}
\usepackage{cases}
\usepackage{enumerate}
\usepackage[ruled,vlined]{algorithm2e}
\usepackage{float}
\usepackage{bm}
\usepackage{psfrag}
\usepackage[hang]{subfigure}

\newtheorem{theorem}{定理}
\newtheorem{lemma}{引理}
\newtheorem{definition}{定义}
\newtheorem{comment}{注}
\newtheorem{conjecture}{Conjecture}

\setlength{\oddsidemargin}{0cm}
\setlength{\evensidemargin}{0cm}
\setlength{\textwidth}{150mm}
\setlength{\textheight}{230mm}

\newcommand\bu{\boldsymbol{u}}
\newcommand\bv{\boldsymbol{v}}
\newcommand\bx{\boldsymbol{x}}
\newcommand\bn{\boldsymbol{n}}
\newcommand\bg{\boldsymbol{g}}
\newcommand\dd{\mathrm{d}}
\newcommand\bbR{\mathbb{R}}
\newcommand\bbV{\mathbb{V}}
\newcommand\bbH{\mathbb{H}}
\renewcommand\div{\mathrm{div}}

\newcommand\pd[2]{\dfrac{\partial {#1}}{\partial {#2}}}
\newcommand\abs[1]{\lvert #1 \rvert}
\newcommand\norm[1]{\lvert\lvert #1 \rvert\rvert}
\newcommand\beps{\bm{\varepsilon}}
\newcommand\bsig{\bm{\sigma}}
\newcommand\bphi{\bm{\phi}}
\newcommand\trace[1]{tr(#1)}

\title{有限元方法上机作业}

\begin{document}
\author{段俊明\thanks{北京大学数学科学学院,科学与工程计算系,邮箱: {\tt duanjm@pku.edu.cn}}
}
\maketitle

\section{问题重述}
先简单介绍三维的弹性力学方程组,
设边界$\partial\Omega$是Lipschitz连续的,位移为$\bu=(u_1,u_2,u_3)^\mathrm{T}$,
应变与位移的关系
\begin{equation}
  \beps(\bu)=\frac12(\nabla\bu+\nabla\bu^{\mathrm{T}}),
\end{equation}
应力与应变的关系
\begin{equation}
  \bsig(\bu)=\lambda\trace{\beps}I+2\mu\beps,
\end{equation}
应变和应力可表示成对称的三阶矩阵,
其中$\lambda,\mu$是拉梅常数,~$\lambda=3.65e4,\lambda+2\mu=6.70e4$.
考虑平衡方程,
\begin{equation}
  -\sum\limits_{j=1}^3 \partial_j\sigma_{ij}(\bu)=f_i,\quad i=1,2,3,
  \label{eq:balance}
\end{equation}
其中$\boldsymbol{f}=(f_1,f_2,f_3)^\mathrm{T}$是体积力,
将应力带入平衡方程,
\begin{align}
  &-\mu\Delta\bu-(\lambda+\mu)\nabla(\nabla\cdot\bu)=\boldsymbol{f},\quad in~\Omega,\\
  &\sum\limits_{j=1}^3 \sigma_{ij}(\bu)\bn_j=g_i,\quad on~\Gamma_1,\\
  &\bu=0,\quad on~\Gamma_0,
\end{align}
其中$\Gamma_0$是固定边界,~$\Gamma_1$是自由边界.

考虑二维平面应变问题,即假设$z$方向应变为0,有
$u_3=0,~\beps_{31}=\beps_{32}=\beps_{33} =0$,
且$\bsig_{31}=\bsig_{32}=0$,
由$\bsig_{33}=\lambda(\beps_{11}+\beps_{22})+2\mu\beps_{33}$
得到$\bsig_{33}=\lambda(\beps_{11}+\beps_{22})$.

求解正方形区域$\Omega=[0,10]\times[0,10]$上的平面弹性力学方程组混合边值问题.
其中,~$\Gamma_0$是正方形的底边,使用固定边界条件,~$\Gamma_1$是正方形的另外三条边,
使用自由边界条件.

\section{数值方法}
取试探函数空间$\bu\in\bbV(0;\Omega)=\{\bu\in\bbH^1(\Omega):\bu|_{\Gamma_0}=0\}$,
及检验函数空间$\bv\in\bbV(0;\Omega)$,
将二维的平衡方程乘以$\bv\in{\bbR^2}$,使用Green公式
\begin{equation}
  \int_{\Omega}u\partial_i v\dd \bx=-\int_{\Omega}\partial_i uv\dd
  \bx+\int_{\Gamma}uv\bn_i\dd s,
\end{equation}
以及边界条件,得到变分形式
\begin{align}
  &a(\bu,\bv) = f(\bv),\quad\forall \bv\in\bbV(0;\Omega)\\
  &a(\bu,\bv) = \int_\Omega \sum_{i,j}\bsig_{ij}(\bu)\beps_{ij}(\bv)\dd\bx, \\
  &f(\bv) = \int_\Omega \boldsymbol{f}\bv\dd \bx + \int_{\Gamma_1}\bg\bv\dd s.
\end{align}

取有限维子空间$\bbV_h(0;\Omega)$逼近原空间,上述变分形式变为
\begin{align}
  &a(\bu_h,\bv_h) = f(\bv_h),\quad\forall \bv_h\in\bbV_h(0;\Omega)\\
  &a(\bu_h,\bv_h) = \int_\Omega \sum_{i,j}\bsig_{ij}(\bu_h)\beps_{ij}(\bv_h)\dd\bx, \\
  &f(\bv_h) = \int_\Omega \boldsymbol{f}\bv_h\dd \bx +
  \int_{\Gamma_1}\bg\bv_h\dd s.
  \label{eq:dis-vp}
\end{align}
取$\bbV_h(0;\Omega)$上的一组基函数$\boldsymbol\bphi_i,\ i=0,\dots,N_h-1$,
令
\begin{equation}
  \bu_h=\sum\limits_{j=0}^{N_h-1}u_j\boldsymbol\bphi_j,~\bv_h=\sum\limits_{j=0}^{N_h-1}v_j\boldsymbol\bphi_j,
\end{equation}
则可以将离散问题\eqref{eq:dis-vp}等价于,求
$\bu_h=(u_0,\dots u_{N_h-1})^\mathrm{T}\in\bbR^{N_h}$,使得
\begin{equation}
  \sum_{i,j=0}^{N_h-1}a(\boldsymbol\bphi_j,\boldsymbol\bphi_i)u_jv_i=\sum_{i=0}^{N_h-1}(f,\boldsymbol\bphi_i)v_i,\quad
  \forall \bv_h=(v_0,\dots,v_{N_h-1})^\mathrm{T}\in\bbR^{N_h},
\end{equation}
由$\bv_h$的任意性,等价于求解线性方程组
\begin{equation}
  \sum_{j=0}^{N_h-1}a(\boldsymbol\bphi_j,\boldsymbol\bphi_i)u_j=(f,\boldsymbol\bphi_i),\quad
  i=0,\dots,N_h-1,
\end{equation}
记为$K\bu_h=\boldsymbol{f}$,~$K=(k_{ij})$.

对区域$\Omega$做三角形剖分,记空间网格尺寸为$h$,记单元为$T_i,\ i=0,\dots,M-1$,节点为$A_i,\
i=0,\dots,N-1$,取分片线性基函数,即$\bphi_{2i},\bphi_{2i+1}\in\bbV_h(0;\Omega)$,满足
\begin{equation}
  \bphi_{2i}(A_j)=(\delta_{ij},0)^\mathrm{T},\quad
  \bphi_{2i+1}(A_j)=(0,\delta_{ij})^\mathrm{T},
  \quad i=0,\dots,N-1, A_i\notin\Gamma_0,
\end{equation}
这里$N_h=2N$.

引入数组$en(\alpha,e)$,其中$e$为单元序数,~$\alpha$为单元的局部基函数序数,即第$e$个
单元的第$\alpha$个基函数对应于第$en(\alpha,e)$个整体节点;
取值为空间坐标的$cd(i,nd)$,其中$nd$为整体节点序数,~$cd(i,nd)$为第$nd$个整体节点
的空间坐标的第$i$个分量.记
$a^e(\bu,\bv)=\int_{T^e}\sum_{i,j}\bsig_{ij}(\bu)\beps_{ij}(\bv)\dd\bx$,
由刚度矩阵的定义,
\begin{equation}
  k_{ij}=a(\bphi_j,\bphi_i)=\sum_{e=0}^{M-1}a^e(\bphi_j,\bphi_i)=\sum_{e=0}^{M-1}k^e_{ij}.
\end{equation}
设$\alpha,\beta=0,\dots,5$为$T^e$的6个基函数的局部序数,即$T^e$上6个基函数分别为
$\bphi^e_0,\dots,A^e_5$,
定义单元刚度矩阵$K^e=(k^e_{\alpha\beta})$,
\begin{equation}
  k^e_{\alpha,\beta}=a^e(\bphi_\alpha^e,\bphi_\beta^e),
\end{equation}
则可以合成为刚度矩阵$K$,
\begin{equation}
  k_{ij}=\sum_{\substack{en(\alpha,e)=i\in T^e\\en(\beta,e)=j\in T^e}} k_{\alpha,\beta}^e.
\end{equation}
同样计算单元载荷向量$\boldsymbol{f}^e=(f^e_\alpha)$,
\begin{equation}
  f^e_\alpha=\int_{T^e}\boldsymbol{f}\bphi_\alpha^e\dd\bx,
\end{equation}
则总载荷向量$\boldsymbol{f}=(f_i)$满足
\begin{equation}
  f_i=\sum\limits_{en(\alpha,e)=i\in T^e}f^e_\alpha.
\end{equation}
生成了刚度矩阵和载荷向量后,我们即可求解线性方程组得到数值解.

下面计算单元刚度矩阵,
设三角形三个顶点坐标为$(x_1,y_1),(x_2,y_2),(x_3,y_3)$,
设点$(x,y)$的重心坐标为$\lambda_1,\lambda_2,\lambda_3$,满足
\begin{equation}
  \begin{bmatrix}
    \lambda_1\\ \lambda_2\\ \lambda_3\\
  \end{bmatrix}
  =
  \dfrac{1}{D}
  \begin{bmatrix}
    a_1 & b_1 & -c_1 \\
    a_2 & b_2 & -c_2 \\
    a_3 & b_3 & -c_3 \\
  \end{bmatrix}
  \begin{bmatrix}
    1 \\ x \\ y\\
  \end{bmatrix},
\end{equation}
其中
\begin{align}
& a_1=x_2y_3-x_3y_2,~b_1=y_2-y_3,~c_1=x_2-x_3, \\
& a_2=x_3y_1-x_1y_3,~b_2=y_3-y_1,~c_2=x_3-x_1, \\
& a_3=x_1y_2-x_2y_1,~b_3=y_1-y_2,~c_3=x_1-x_2.
\end{align}
$D$为三角形面积的2倍.
设位移为
$u=u_1\lambda_1+u_2\lambda_2+u_3\lambda_3,v=v_1\lambda_1+v_2\lambda_2+v_3\lambda_3$,
则应变满足
\begin{equation}
  \beps=
  \begin{bmatrix}
    \varepsilon_x \\
    \varepsilon_y \\
    \varepsilon_{xy}\\
  \end{bmatrix}
  =
  \begin{bmatrix}
    \pd{u}{x} \\
    \pd{v}{y} \\
    \pd{u}{y}+\pd{v}{x} \\
  \end{bmatrix}
    =
    \begin{bmatrix}
      \pd{\lambda_1}{x} & 0 & \pd{\lambda_2}{x} & 0 & \pd{\lambda_3}{x} & 0 \\
      0 & \pd{\lambda_1}{y} & 0 & \pd{\lambda_2}{y} & 0 & \pd{\lambda_3}{y} \\
      \pd{\lambda_1}{y} & \pd{\lambda_1}{x} & \pd{\lambda_2}{y} & \pd{\lambda_2}{x} & \pd{\lambda_3}{y} & \pd{\lambda_3}{x} \\
    \end{bmatrix}
    \begin{bmatrix}
      u_1 \\ v_1 \\ u_2 \\ v_2 \\ u_3 \\ v_3 \\
    \end{bmatrix}
    \triangleq B\tilde{\bm{u}},
\end{equation}
计算得
\begin{equation}
  B=
  \dfrac{1}{D}
  \begin{bmatrix}
    b_1 & 0 & b_2 & 0 & b_3 & 0 \\
    0 & -c_1 & 0 & -c_2 & 0 & -c_3 \\
    -c_1 & b_1 & -c_2 & b_2 & -c_3 & b_3 \\
  \end{bmatrix},
\end{equation}
注意这里$\varepsilon_{xy}=2\beps_{12}$,但最后推导出的单元刚度矩阵相同.
应力满足
\begin{equation}
  \bsig=\begin{bmatrix}
    {\sigma_x} \\
    {\sigma_y} \\
    \sigma_{xy}\\
    \end{bmatrix}
    =
    \begin{bmatrix}
      \lambda+2\mu & \lambda & 0 \\
      \lambda & \lambda+2\mu & 0 \\
      0 & 0 & \mu \\
    \end{bmatrix}
    \beps
    \triangleq L\beps,
\end{equation}
则
\begin{equation}
  K^e=\int_{T^e}\bsig^{\mathrm{T}}\beps\dd\bx=\dfrac{D}{2}B^\mathrm{T}LB.
\end{equation}

单元载荷向量我们通过数值积分来计算,这里单元上的数值积分使用2阶代数精度的三角形上
的Gauss积分点,边界上的线积分使用3阶代数精度的一维的Gauss积分点.

\section{数值实验}
令体积力$f_1=f_2=0$,沿法向受力为$0$,即$g_1=g_2=0$,除了左上角的
第一个网格,它的受力为$\sigma_x=1,\sigma_y=-1,\sigma_{xy}=0$,则它的两条边的边界条件为
$$g_{l1}=-1,$$
$$g_{l2}=0,$$
$$g_{u1}=0, $$
$$g_{u2}=-1, $$
下标$l,r,u$分别表示左、右、上边界,下标$1,2$分别表示第一、二个分量.


\end{document}

