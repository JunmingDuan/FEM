\documentclass[a3paper, 11pt]{ctexart}
\usepackage{srcltx,graphicx}
\usepackage{amsmath, amssymb, amsthm}
\usepackage{color}
\usepackage{lscape}
\usepackage{multirow}
\usepackage{cases}
\usepackage{enumerate}
\usepackage[ruled,vlined]{algorithm2e}
\usepackage{float}
\usepackage{bm}
\usepackage{psfrag}
\usepackage[hang]{subfigure}

\newtheorem{theorem}{定理}
\newtheorem{lemma}{引理}
\newtheorem{definition}{定义}
\newtheorem{comment}{注}
\newtheorem{conjecture}{Conjecture}

\setlength{\oddsidemargin}{0cm}
\setlength{\evensidemargin}{0cm}
\setlength{\textwidth}{150mm}
\setlength{\textheight}{230mm}

\newcommand\bu{\boldsymbol{u}}
\newcommand\bn{\boldsymbol{n}}
\newcommand\dd{\mathrm{d}}
\renewcommand\div{\mathrm{div}}

\newcommand\diff{\,\mathrm{d}}

\newcommand\pd[2]{\dfrac{\partial {#1}}{\partial {#2}}}
\newcommand\abs[1]{\lvert #1 \rvert}
\newcommand\norm[1]{\lvert\lvert #1 \rvert\rvert}
\newcommand\beps{\boldsymbol{\varepsilon}}
\newcommand\bsig{\boldsymbol{\sigma}}
\newcommand\trace[1]{tr(#1)}

\title{有限元方法上机作业}
\author{段俊明\thanks{北京大学数学科学学院,科学与工程计算系,邮箱: {\tt duanjm@pku.edu.cn}}
}

\begin{document}
\maketitle

\section{问题重述}
求解正方形区域$\Omega=[0,10]\times[0,10]$上的二维弹性力学方程组混合边值问题.
设边界$\partial\Omega$是Lipschitz连续的,位移为$\bu=(u,v)^\mathrm{T}$,
应变与位移的关系
\begin{equation}
  \beps(\bu)=\frac12(\nabla\bu+\nabla\bu^{\mathrm{T}}),
\end{equation}
应力与应变的关系
\begin{equation}
  \bsig(\bu)=\lambda\trace{\beps}I+2\mu\beps,
\end{equation}
其中$\lambda,\mu$是拉梅常数,~$\lambda=3.65e4,\lambda+2\mu=6.70e4$.
考虑平衡方程,
\begin{equation}
  -\sum\limits_{j=1}^2 \partial_j\sigma_{ij}(\bu)=f_i,\quad i=1,2,\\
\end{equation}
其中$\boldsymbol{f}=(f_1,f_2)^\mathrm{T}$是外力,
将应力带入平衡方程,
\begin{align}
  &-\mu\Delta\bu-(\lambda+\mu)\nabla(\nabla\cdot\bu)=\boldsymbol{f},\quad in~\Omega,\\
  &\sum\limits_{j=1}^2 \sigma_{ij}(\bu)\bn_j=g_i,\quad on~\Gamma_1,\\
  &\bu=0,\quad on~\Gamma_0,
\end{align}
其中,~$\Gamma_0$是正方形的底边,使用固定边界条件,~$\Gamma_1$是正方形的另外三条边,
使用自由边界条件.

假设精确解为$u=x^2y^2,v=x^2\sin(y)$,带入以上方程组得到,
$$f_1=-\left(2\,y^2+2\,x\,\cos y\right)\,\left(\mu+\lambda\right)-\mu\,
\left(2\,x^2+2\,y^2\right),$$
$$f_2=-\left(4\,x\,y-x^2\,\sin y\right)\,\left(\mu+\lambda\right)-\mu\,
\left(2\,\sin y-x^2\,\sin y\right),$$
$$g_{l1}=-\lambda(x^2\cos y+2xy^2)-4\mu xy^2, $$
$$g_{l2}=-2\mu(x\sin y+x^2y), $$
$$g_{r1}=\lambda(x^2\cos y+2xy^2)+4\mu xy^2, $$
$$g_{r2}=2\mu(x\sin y+x^2y), $$
$$g_{u1}=2\mu(x\sin y+x^2y), $$
$$g_{u2}=\lambda(x^2\cos y+2xy^2)+2\mu x^2\cos y, $$
下标$l,r,u$分别表示左、右、上边界,下标$1,2$分别表示第一、二个分量.

\section{数值方法}
将原问题化为变分形式:
%\begin{equation}
	%\begin{cases}
		%a(u,v) = f(v), \\
		%a(u,v) = \int_\Omega \nabla u·\nabla v\dd \bx + \beta\int_{\Omega_1}uv\dd s, \\
		%f(v) = \int_\Omega fv\dd \bx + \int_{\Omega_1}gv\dd s, \\
		%u\in\bbV_1=\{\bbH^1(\Omega),u|_{\partial\Omega_0 = u_0} \}, \\
		%v\in\bbV_2=\{\bbH^1(\Omega),v|_{\partial\Omega_0 = 0} \}. \\
	%\end{cases}
%\end{equation}

%取xy方向同样为$N$的网格数,将以上变分问题离散化,等价与求解线性方程组:
%\begin{equation}
	%\sum_{i=1}^{N_h}a(\phi_j,\phi_i)u_j=(f,\phi_i),\quad i=1,2,\dots,N_h.
%\end{equation}
%其中$N_h$为总节点数,$\phi_i$为一组基函数,在第$i$个节点上为1,其它节点处为0.
%$u_i$为第$i$个节点上的数值解.

%将上述方程组写作:
%$$\boldsymbol K\bu_h=\boldsymbol f,$$
%并称$\boldsymbol K$为刚度矩阵,$\bu_h$为位移向量,$\boldsymbol f$为载荷向量.

%下面需要确定的就是$\boldsymbol K$和$\boldsymbol f$的元素.

%用$e$来表示单元序号,$T_e$表示单元,$a^e(u,v)=\int_{T_e}\nabla u·\nabla v\dd \bx$,
%定义单元刚度矩阵$\boldsymbol K^e$和单元载荷向量$\boldsymbol f^e$,
%则
%\begin{align}
	%k_{ij}=a(\phi_j,\phi_i)=\sum_{e=1}^Ma^e(\phi_j,\phi_i)=\sum_{e=1}^Mk_{ij}^e.
%\end{align}

%其中$k_{ij}^e$表示第$e$个单元的单元刚度矩阵的$(i,j)$元素.
%对于型(1)三角形单元,书上$P_{213}$已经给出单元刚度矩阵$\boldsymbol K^e$的结果;
%对于型(1)矩形单元,下面给出单元刚度矩阵:

%取边长为1的正方形,有四个节点$(0,0),(1,0),(1,1),(0,1)$,给出它们的一组基:
%$$(1-x)(1-y),x(1-y),xy,(1-x)y.$$
%利用与书上类似的仿射变换,将任意正方形上的单元刚度矩阵统一到标准正方形上来处理,
%最终得到:
%\[ \boldsymbol K^e=\dfrac{h^2}{\det A_e}\begin{bmatrix}
		%\frac 23 & -\frac 16 && -\frac 13 && -\frac 16 \\
		%-\frac 16 & \frac 23 && -\frac 16 && -\frac 13 \\
		%-\frac 13 & -\frac 16 && \frac 23 && -\frac 16 \\
		%-\frac 16 & -\frac 13 && -\frac 16 && \frac 23 \\
	%\end{bmatrix} \]

%当$f$为常数时,与书上类似得到:
%\[ \boldsymbol f^e=\dfrac 14f(T_e)\abs{T_e}(1,1,1)^\mathrm T. \]

%对于边界项积分,采用数值积分公式,并将其加入到刚度矩阵和载荷向量中;
%对于Diriclet边界上的节点,将其从生成的线性方程组中去除.

%\section{算法}
%基于以上分析,有如下算法:

%\begin{algorithm}[H]
%\caption{生成刚度矩阵和载荷向量}
%\SetAlgoNoLine
%$\boldsymbol K= (k(i,j))=0;\boldsymbol f=(f(i))=0;$\\
%\For{$e=1:M$}
%{
	%$\text{计算单元刚度矩阵}\boldsymbol K^e=(k^e(\alpha,\beta))$; \\
	%$\text{计算单元载荷向量}\boldsymbol f^e=(f^e(\alpha))$; \\
	%$k^e(en(\alpha,e),en(\beta,e))+=k^e(\alpha,\beta)$; \\
	%$f^e(en(\alpha,e))+=f^e(\alpha)$; \\
%}
%\label{al:fem}
%\end{algorithm}

%其中$en(\alpha,e)$是取值为整体节点序数的数组,$\alpha$为局部节点序数.
%还需要保存单元坐标,$cd(i,nd)$是取值为空间坐标的数组,表示第$nd$个整体节点
%的空间坐标的第$i$个分量.

%\section{程序说明}
%main.cpp是调用fem.h的主函数,space.h是计算cd,en两个数组的头文件,
%fem.cpp实现了该算例生成线性方程组并求解的过程.
%线性方程组部分调用了Eigen的稀疏矩阵LU求解器.


\end{document}

